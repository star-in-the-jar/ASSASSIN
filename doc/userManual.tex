\documentclass{documentation}

\title{Instrukcja użytkownika systemu ASSASSIN}

\author{Tomasz Chady}

\begin{document}

\maketitle

\tableofcontents

\section{Pacjent}

W systemie ASSASSIN pacjent ma możliwość przeglądania swoich wyników badań.
Na tym się kończy jego rola w systemie.

\subsection{Tworzenie konta}

Tworzenie konta to nie jest odpowiedzialność pacjenta.
Konto jest tworzone przy pierwszym kontakcie pacjenta z systemem przez lekarza lub inny personel.
Zatem jeśli pacjent nie ma konta, powinien skontaktować się z lekarzem.
Przy tworzeniu konta personel medyczny powinien podać pacjentowi login i hasło.

\subsection{Logowanie}

Po wejściu na stronę główną systemu, pacjent powinien kliknąć przycisk \textit{Logowanie pacjenta}.
Powinien pojawić się wtedy formularz logowania.
Następnie pacjent powinien wpisać swój login i hasło, a następnie kliknąć przycisk \textit{Zaloguj}.
W przypadku niepowodzenia, pacjent powinien skontaktować się z personelem medycznym w celu resetowania loginu i hasła.
Jeśli aktywowane jest logowanie dwuetapowe, pojawi się okno wpisywania kodu 2FA.
Po wpisaniu kodu pacjent powinien być przekierowany do strony głównej systemu.

\subsection{Odbieranie wyników}

Po zalogowaniu pacjent powinien zobaczyć panel do pobierania wyników badań.
Po wpisaniu informacji identyfikujących badanie pacjent powinien zostać zabrany do strony z wynikami.
Na stronie z wynikami z lewej strony powinien powinny pojawić się dwa panele: panel z informacjami o badaniu, oraz panel inteligentnego asystenta.
Z prawej strony powinien pojawić się panel z wynikami badań.
Przez kliknięcie przycisku \textit{Pobierz} pacjent powinien pobrać plik z wynikami badań w postaci pliku PDF.

\section{Lekarz}



\end{document}