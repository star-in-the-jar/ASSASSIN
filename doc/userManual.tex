\documentclass{documentation}

\title{Instrukcja użytkownika systemu ASSASSIN}

\author{Tomasz Chady}

\begin{document}

\maketitle

\tableofcontents

\HUGEskip

\section{Wstęp}

System ASSASSIN jest przeznaczony dla szpitali, klinik i laboratoriów.
Ta instrukcja powinna być używana przez personel medyczny oraz pacjentów.
Częścią wspólną systemu jest strona główna, która jest przedstawiona na schemacie \ref{fig:mainPage}.

\begin{figure}[h]
    \centering
    \includegraphics[width=0.8\textwidth]{landing page.png}
    \caption{Strona główna systemu\label{fig:mainPage}}
\end{figure}

W lewym górnym rogu znajduje się przycisk prowadzący do logowania personelu medycznego.
Z kolei w prawym górnym rogu znajduje się przycisk prowadzący do logowania pacjenta.

\section{Weryfikacja dwuetapowa}

Weryfikacja dwuetapowa odbywa się na zasadzie mechanizmu \href{https://pl.wikipedia.org/wiki/Has%C5%82o_jednorazowe}{OTP}.
Mechanizm OTP(one time password) działa na zasadzie wspólnego klucza pomiędzy serwerem a aplikacją autentykacyjną.
Klucz jest wykorzystywany do generowania kodów jednorazowych, które tracą ważność co 30 sekund.
Aplikacje autentykacyjne, które generują klucze jednorazowe na podstawie wspólnego klucza są dostępne na wszystkie popularne platformy mobilne.
Polecamy aplikację \href{https://play.google.com/store/apps/details?id=com.google.android.apps.authenticator2&hl=pl}{Google Authenticator}.
Po zalogowaniu się do systemu, użytkownik powinien wpisać kod wygenerowany przez aplikację autentykacyjną.
Weryfikacja dwuetapowa jest opcjonalna i może być włączona lub wyłączona przez administratora systemu.

\section{Pacjent}

W systemie ASSASSIN pacjent ma możliwość przeglądania swoich wyników badań.
Na tym się kończy jego rola w systemie.
Pacjent nie ma możliwości przeglądania wyników badań innych pacjentów.

\subsection{Tworzenie konta}

Tworzenie konta to nie jest odpowiedzialność pacjenta.
Konto jest tworzone przy pierwszym kontakcie pacjenta z systemem przez lekarza lub inny personel.
Zatem jeśli pacjent nie ma konta, powinien skontaktować się z lekarzem.
Przy tworzeniu konta personel medyczny powinien podać pacjentowi login i hasło.

\subsection{Logowanie}

Po wejściu na stronę główną systemu, pacjent powinien kliknąć przycisk \textit{Logowanie pacjenta}.
Powinien pojawić się wtedy formularz logowania.
Wygląd formularza logowania jest przedstawiony na schemacie \ref{fig:login}.

\begin{figure}[h]
    \centering
    \includegraphics[width=0.8\textwidth]{logowanie.png}
    \caption{Widok logowania\label{fig:login}}
\end{figure}

Następnie pacjent powinien wpisać swój login i hasło, a następnie kliknąć przycisk \textit{Zaloguj się}.
W przypadku niepowodzenia, pacjent powinien skontaktować się z personelem medycznym w celu resetowania loginu i hasła.
Jeśli aktywowane jest logowanie dwuetapowe, pojawi się okno wpisywania kodu 2FA.
Przykładowy widok weryfikacji dwu stopniowej jest przedstawiony na schemacie \ref{fig:2FA}.

\begin{figure}[h]
    \centering
    \includegraphics[width=0.8\textwidth]{2FA on log.png}
    \caption{Widok weryfikacji dwu stopniowej\label{fig:2FA}}
\end{figure}

Po wpisaniu kodu proces autentykacji jest zakończony i pacjent powinien zostać zabrany do strony profilowej.
Ze strony profilowej pacjent powinien mieć możliwość przejścia do strony z wynikami badań.
Strona profilowa jest przedstawiona na schemacie \ref{fig:patientProfile}.

\begin{figure}[h]
    \centering
    \includegraphics[width=0.8\textwidth]{logged in.png}
    \caption{Widok profilu pacjenta\label{fig:patientProfile}}
\end{figure}

\subsection{Odbieranie wyników}

Po kliknięciu na przycisk \textit{Przejdź do pobierania wyników} pacjent powinien zostać zabrany do widoku wprowadzania danych identyfikujących.
Widok wprowadzania danych identyfikujących badanie jest przedstawiony na schemacie \ref{fig:patientData}.

\begin{figure}[h]
    \centering
    \includegraphics[width=0.8\textwidth]{pobieranie-wyników.png}
    \caption{Widok pobierania wyników\label{fig:patientData}}
\end{figure}

Po wpisaniu informacji identyfikujących badanie pacjent powinien zostać zabrany do strony z wynikami.
Na stronie z wynikami z lewej strony powinien powinny pojawić się dwa panele: panel z informacjami o badaniu, oraz panel inteligentnego asystenta.
Z prawej strony powinien pojawić się panel z wynikami badań.
Widok wyników badań jest przedstawiony na schemacie \ref{fig:results}.

\begin{figure}[h]
    \centering
    \includegraphics[width=0.8\textwidth]{wyniki-szpital.png}
    \caption{Widok pobierania wyników\label{fig:results}}
\end{figure}

Przez kliknięcie przycisku \textit{Pobierz} pacjent powinien pobrać plik z wynikami badań w postaci pliku PDF.

\section{Personel medyczny}

W odróżnieniu od pacjentów personel medyczny nie ma osobnych kont.

\subsection{Logowanie}

Autoryzacja personelu następuje przez placówkę medyczną.
W momencie logowania personel medyczny podaje login szpitala, hasło oraz kod 2FA.
W przypadku niepowodzenia personel medyczny powinien skontaktować się z administratorem systemu.
Po zalogowaniu personel medyczny powinien zostać zabrany do strony profilowej.

\subsection{Dostęp do wyników}

Strona profilowa personelu medycznego jest bardzo podobna do strony profilowej pacjenta.
Różnica polega na tym, że personel medyczny ma dostęp do wyników badań wszystkich pacjentów swojej placówki medycznej.
Wystarczy jedynie podać dane pacjenta w widoku wprowadzania danych identyfikujących badanie.
Po podaniu danych identyfikujących badanie personel medyczny powinien zostać zabrany do strony z wynikami.
Widok wyników badań jest identyczny jak w przypadku pacjenta.

\end{document}